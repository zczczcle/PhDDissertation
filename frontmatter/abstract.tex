% the abstract
\justify

\pagenumbering{roman}
\setcounter{page}{5}

Fast development of Transmission Electron Microscopy (TEM) is pushing boundaries for nanomaterial observations at the unprecedented high spatial resolution, down to 60 pm, these days. The advances of sampling techniques, lens aberration corrections and spectroscopic analysis allow for entire understanding of various materials atomic structures and spatially-resolved chemical compositions. However, common TEM techniques have had no access to nanomaterial electrical, mechanical, optical and thermal properties, which may be advantageous for future applications, such as flexible electronics, optoelectronics, green energy storage etc. Thus it is crucial to find a way to manipulate, contact, and \textit{in situ} probe a nanomaterial in order to reveal its peculiar functionality. For example, in order to understand light-matter interactions, electrical and optoelectronic properties of semiconducting nanomaterials and their heterostructures, it is essential to perform challenging optoelectronic tests under various manipulations inside a high-resolution TEM. The beauty and power of the state-of-the-art \textit{in situ} TEM experiments stem from a fact that any functional property may be measured under a full control of the nanomaterial atomic structure, its defects and chemistry. This allows for the unambiguous establishment of the clear structure-property relationship. And this is the “Holy Grail” of the whole Materials Science field. However, this task is never as simple as handling and assembling toy bricks, especially inside a TEM. Throughout my PhD term I was able to successfully perform diverse \textit{in situ} TEM probing experiments on a rich bunch of nanomaterials which will be used in the future lithium/sodium ion batteries, flexible electronic and optoelectronic devices, and, overall, for any desired \textit{nanoarchitectonics} concept. \\
This Thesis comprises of 7 Chapters. Chapter 1 is an outline of the background of the probing inside TEM, which presents the development of \textit{in situ} TEM techniques. The content shows a novelty of {\it in situ} TEM, its advantages and a scope of nanomaterials which are under study. In Chapter 2, methods and engineering details of my research are presented, including {\it in situ} TEM setups and its applications. Chapter 3 is the first experimental Chapter introducing manipulation possibilities in the frame of the general {\it nanoarchitechtonics} concept and its applications for nanoengineering. The experiments have been performed on the {\it in situ} TEM constructed axial nanowire junctions of CdS and p-Si. Then, detailed electrical probing for energy storage research is discussed in Chapter 4. We fabricated an ultra-stable sodium ion battery and analyzed the mechanism of its cycling performance under {\it in situ} TEM probing. Through coupling with {\it in situ} TEM applied forces, two examples of force-driven optoelectronic phenomena are detailed in Chapters 5 and 6. Finally, in Chapter 7, the whole research carried out through the Thesis is reviewed and general conclusions are drawn. Also in this final Chapter, suggestions are made for the future work in this booming field. 
