% the abstract
\justify

Fast development of Transmission Electron Microscopy (TEM) is pushing boundaries for nanomaterial observations at the unprecedented high spatial resolution, down to 60 pm, these days. The advances of sampling techniques, lens aberration corrections and spectroscopic analysis allow for entire understanding of various materials atomic structures and spatially-resolved chemical compositions. However, common TEM techniques have had no access to nanomaterial electrical, mechanical, optical and thermal properties which may be advantageous for future applications, such as flexible electronics, optoelectronics, green energy storage etc. Thus it is crucial to find a way to manipulate, contact, and in situ probe a nanomaterial in order to reveal its peculiar functionality. For example, in order to understand light-matter interactions, electrical and optoelectronic properties of semiconducting nanomaterials and their heterostructures, it is essential to perform challenging optoelectronic tests under various manipulations inside a high-resolution TEM. The beauty and power of the state-of-the-art in situ TEM experiments stem from a fact that any functional property may be measured under a full control of the nanomaterial atomic structure, its defects and chemistry and allow for the unambiguous establishment of the clear structure-property relationship. And this is the “Holy Grail” of the whole Materials Science field. However, this task is never as simple as handling and building toy bricks, especially inside a TEM. Throughout my PhD term I was able to successfully perform diverse in situ TEM probing experiments on a rich bunch of nanomaterials which will be used in the future lithium/sodium ion batteries, flexible electronic and optoelectronic devices, and, overall, for any desired nanoarchitectonics concept. \\
This Thesis comprises of 6 parts. Part 1 is an outline of the background of the probing inside TEM, which presents the development of in situ TEM techniques. The content shows a novelty of in situ TEM, its advantages and a scope of nanomaterials which is under study. In Part 2, experimental methods and engineering details of my research are presented, including materials synthesis, in situ TEM setups and device fabrications. Schematic of the experimental setup is shown in Figure 1. Part 3 is the first experimental Chapter introducing manipulation possibilities in the frame of the general {\em nanoarchitechtonics} concept and its applications for nanoengineering. The experiments have been performed on the in situ TEM constructed axial nanowire junctions of CdS and p-Si. Then, detailed electrical probing for energy storage research is discussed in Part 4. We fabricated an ultra-stable sodium ion battery and analyzed the mechanism of its cycling performance under in situ TEM probing. Through coupling with in situ TEM applied mechanical forces, two examples of force driven optoelectronic phenomena are detailed in Part 5. Finally, in Chapter 6, the whole work carried out through the Thesis is reviewed and general conclusions are drawn. Also in this final Chapter, suggestions are made for the future work in this booming field. 
