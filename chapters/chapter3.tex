%update: Jan 15 fixed figure/table numbers, fixed figure captions.  
%update: Jan 14 fixed grammar according to prof notes
%update: Jan 13 prof rewrite for ithenticate
%update: Jan 11 table 3.1 fixed
%update: Jan 09-11 prof check
%update: Jan 03 table ok. 
%update: Nov 21 fixed equation part. 
%update: Nov 09 by professor, rewrote all text. 

%\begin{savequote}[75mm] 
%It's rather easy to play any musical instrument: all you have to do is to touch the right key at the right time and the instrument will play itself.
%\qauthor{Johann Sebastian Bach} 
%\end{savequote}

\chapter{Nanoarchitechtonics of Axial Nanowire Junctions of CdS and p-Si}

\newthought{A high-precision technique} was implemented to build and analyze axial nanowire heterojunctions inside a high-resolution transmission electron microscope (HRTEM). Through {\em in-tandem} using of a sharp tungsten probe as the nanomanipulator and an optical fiber as the optical waveguide the nanoscale CdS/p-Si axial nanowire junctions were constructed, and \textit{in situ} recorded photocurrents from them were detected. Compared to the individual constituting nanowires, the CdS/p-Si axial nanowire junctions exhibit a photocurrent saturation effect which protects them from damage under high voltages. In addition, a set of experiments demonstrates the clear relationship between the saturation photocurrent values and the incident light intensities. The applied technique is envisaged to be valuable for bottom-up nanodevice fabrications, and the documented photocurrent saturation feature should solve the Joule heating-induced failure problem in nanowire optoelectronic devices caused by a fluctuating bias. 

\section{Introduction}
These days, the the key progresses in nanoscale photonics and electronics are made thaks to the significant improvement of functional device performances. Nanowires, as prime building blocks in the bottom up technology, have been shown to be the key candidates for the next generation booming nanophotonic applications because of their good crystallinity, high carrier mobility, confinement effects and infinite possibilities of assembling any required architecture for diverse utilizations \cite{lieberprogramable2014,tsai2014,zhangx2014}. However, till now, fabrication of a desired nanoarchitecture employing nanoscale building blocks has been a challenge. Several nanowire-based devices, such as transistors \cite{577926446,577926447,577926448}, diodes \cite{577926449}, photodetectors \cite{577926451,577926452} and logic circuits \cite{577926453,577926454}, have successfully been constructed on substrates through diverse lithography techniques. By contrast, controlled manipulation with two or more individual objects with a nanoscale precision and on-site creation of axial heteroarchitectures made of them (for the immediate optoelectronic probing) has never been tried. Building of the regarded junctions and \textit{in situ} testing of their optoelectronic characteristics would be highly important in relation to the “nanoarhitectonics” concept and uncovering novel physical properties and phenomena.\\ 

Cadmium sulfide (CdS) is known as one of the key materials in heterojunction type solar cell because of its advantageous type II window band structure  \cite{577926455}. Also, it was shown that merging of CdS and Si materials results in a decent junction. Therefore, numerous new functions and  utilizations may be envisioned. In addition, a careful study revealed that CdS/p-type-Si junctions are basically better than CdS/n-type-Si junctions for rectifying properties because of their specific type II band structure \cite{577926457}. Nevertheless, reliable usage of these two "hot" optical materials, i.e. CdS and Si, is rather rare, because both junction constituents are not transparent to a solar light. And the normal layered structures are considered not to be efficient. In order to directly expose the heterojunctions to the light, a smart way is to build the nanowire array structures \cite{577926458,577926459}. In addition to the wide-spread core-shell nanowire ensembles, constructing axial nanowire heterojunctions by means of two semiconducting materials is a promising experimental route.\\

\begin{figure}  
\includegraphics[width=\textwidth]{figures/figure3_s1}
\caption[SEM and TEM of CdS nanowires.]{(a) SEM image of CdS nanowires. (b) TEM image and corresponding fast Fourier
transform pattern of an individual CdS nanowire. 
\label{fig:fig3_s1}}
\end{figure}

Following previously made axial nanowire heterojunctions for diverse optoelectronic applications, CdS nanowires and B-doped Si nanowires have been selected by me as the targeting building blocks. Thus, in this Chapter, I demonstrate an accurate nanomanipulation technique pioneered in a HRTEM for building new axial nanowire architectures. Straightforward \emph{in situ} electronic and optoelectronic tests are then carried out on them using the light of various wavelengths shining into the TEM column. The designed experiments allow me to simultaneously have an entire control over the crystallography and spatially-resolved chemistry of the two constituting domains and their interfacial region before, during and after optoelectronic probing with high spatial and temporal resolutions specifically achievable with HRTEM. 
My experiments reveal clear photosensing properties of the axial CdS/p-Si nanowire junctions. The latter demonstrate selective sensitivity to blue and purple lights rather than to the light of larger wavelengths. Also, the junctions display a photocurrent saturation effect. This implies that such junctions are applicable in detecting light intensity due to their low energy consumption and stability under unexpectedly pulsing biases. 

\section{Experimental}
\subsection{Material Synthesis}

\begin{figure}  
\includegraphics[width=\textwidth]{figures/figure3_1}
\caption[Making an axial junction.]{\textit{In situ} TEM images representing the fabrication process of an axial CdS/p-Si nanowire junction under manipulation in the electron microscope. (a) Making physical contact of a piezo-driven sharp tungsten probe with an individual clean CdS nanowire on a gold support during the first step of the manipulation; the inset shows soldering the tungsten probe and the wire under focused electron beam irradiation. (b) Pulling away the nanowire from the gold support; (c) connecting a CdS nanowire to the individual B-doped Si nanowire during the second step of the manipulation. The incoming light shining on the junction is marked with an arrow.
\label{fig:fig3_1}}
\end{figure}

The CdS nanowires were prepared using an Au-catalyzed vapor-liquid-solid (VLS) growth  in a chemical vapor deposition (CVD) system, which is analogous to that used in many works \cite{zhang2014photosensing,577926461}. 1 gram of a CdS powder (99.995\%) was put on a graphite plate at the tube furnace center as the source material. A (100) Si wafer covered with a 10 nm thick Au layer was put on the other graphite plate located downstream, at a distance of 11.5 cm from the tube center. The tube was purged under nitrogen flow at 200 °C for 2 h, and then heated to 1000 °C at a rate of 30 °C per min. After 30 min of reaction, the furnace was naturally cooled down to room temperature. The process took place under a \ce{N2} flow of 300 sccm. A wool-like yellowish product was found on the Si substrate after cooling. 
Si nanowires were fabricated via VLS mechanism in a separate CVD system. Au particles of 3 nm in diameter were taken as a metal catalyst. The B-doped nanowires were directly prepared onto Au-coated (111) Si substrates at 600°C for 30 min in a flowing 19 sccm of \ce{SiH4} as a Si reactant gas, and diborane (\ce{B2H6}) was employed as a B precursor. The \ce{B2H6} flow in \ce{H2} was 0.2 sccm and 30 sccm of \ce{N2} served as the carrier gas. Other details on B-doped Si nanowires characterizations were presented in the literature. \cite{577926462,577926464,577926465}.

\begin{figure}  
\includegraphics[width=\textwidth]{figures/figure3_2}
\caption[HRTEM anaysis on junction.]{(a) General HRTEM image of the interfacial area of the created CdS-p-Si junction; (b)–(e) HRTEM images taken in the areas marked in (a) showing the single-crystalline character of both nanowire branches, and (f), (g) corresponding FFT pattern and selected area electron diffraction pattern (SAED) taken from the interfacial region. Characteristic crystallographic directions (b)–(e) and reflections (f), (g) are marked.
\label{fig:fig3_2}}
\end{figure}

\subsection{Techniques}
Field-emission scanning electron microscopy (FE-SEM) of the prepared nanostructures was performed on a Hitachi S-4800 FE-SEM operated at 10 kV. HRTEM analysis and {\em in situ} experiments were carried out using an optimized piezo-driven optical TEM holder, which is discribed in detail in Chapter 2, in an energy-filtering 300 kV JEM 3100FEF (Omega Filter) high-resolution TEM. The multimode fiber (Nanonics Imaging, Ltd.) was threaded through the holder inner channel. The fiber was connected to four laser diodes, with 405, 488, 638 and 808 nm wavelengths, and a tunable power and temperature (Thorlabs, Inc.) were used. \\
The working temperatures of laser diodes were set at 30°C. Firstly, the numerous CdS nanowires were placed onto a fresh-cut flattened Au tip covered with an electrically conductive Ag epoxy under the flash tip immersing into the CdS nanopowder sample. After heating the paint, the Au tip with the specimen was placed within the sample holder. The ultrasharp W probes used as counter-electrodes and manipulators were prepared under NaOH electrochemical etching. The W tip movements were controlled inside TEM in 3 dimensions using a piezoelectric motor for making a contact, and to test and retract the selected CdS nanowires which had been conveniently oriented with respect to the manipulator. Then the fabrication of axial CdS/p-Si nanowire junctions was gently performed in two steps, as described in the following section. Typically, prior to contacting the two nanowire building blocks, an electron beam was applied to focus on the tip-ends of both CdS and p-Si nanowires for 30 s to clean the surfaces. The current–voltage (I–V) curves were recorded by using a Keithley 2612B sourcemeter. The electron beam was typically turned off during the electrical and optoelectronic tests. 


\section{Results and discussions}
As depicted in SEM image of Figure \ref{fig:fig3_s1}a, prepared CdS nanowires, $>50$ μm long, were evenly distributed over a Si substrate over a large area. In Figure \ref{fig:fig3_s1}b, a high-resolution TEM image and the fast Fourier transform pattern (FFT) confirm that an individual nanowire has a well-crystallized hexagonal structure. The growth direction is parallel to the $c$-axis and the lattice constant $c = 0.672$ nm. 

\begin{figure}  
\includegraphics[width=\textwidth]{figures/figure3_3}
\caption[Photocurrent through junction.]{A photocurrent response of an individual CdS/p-Si axial nanowire junction under a dark condition and during illumination with the light of various wavelengths and with fixed intensity. The inset is the low-magnification TEM image of the CdS nanowire structure under testing. The colored arrow shows the incoming light incidence; the black arrow marks a short Si-branch segment.
\label{fig:fig3_3}}
\end{figure}

Because the deoxidized Si nanowires are less stable than CdS nanowires in air, for constructing heterojunctions, it was important to transfer a CdS nanowire into the HRTEM first, and then to contact a Si nanowire, not \textit{vice versa}. As shown in Figure \ref{fig:fig3_1}a, the sharp W probe was precisely manipulated inside HRTEM to contact a pre-selected clean individual CdS nanowire on the Au stage. To pull out the CdS nanowire from the sample stage, an electron-beam soldering, i.e. "glue" technique, was utilized for making the tight contact between the probe and the nanowire. \\

As illustrated in the inset of Figure \ref{fig:fig3_1}a, by focusing a convergent electron probe (about 500 nm in diameter) on the contact area for 20 min, nearly 75 nm thick layer of the residual amorphous carbon (always present in the TEM chamber from lenses, apertures etc.) formed on the probe/nanowire surfaces  resulting in their intimate nano-soldering. \\

Then, the targeted CdS nanowire was detached from the Au sample tip under delicate pulling back the W probe, as shown in Figure \ref{fig:fig3_1}b. To build a final heterojunction, the second step was to contact the pulled-out CdS nanowire with an individual pre-selected boron-doped-Si nanowire. To do so, after retracting the sample holder (with the regarded CdS nanowire on the tungsten probe) from the microscope, the gold sample stage with CdS nanowires was replaced in HRTEM by the fresh Au sample stage with attached numerous deoxidized B-doped Si nanowires. By precisely approaching and contacting the end of the CdS nanowire probe to a selected Si nanowire tip, the axial heterojunction architectures were formed. Figure \ref{fig:fig3_1}c depicts a typical CdS/p-Si axial nanowire junction. These representative CdS and Si nanowires have diameters of ~187 nm and ~46 nm, respectively. The electron beam intensity was immediately weakened after the junction had been constructed. This was done to avoid the nanostructure overexposure to the electron beam which would lead to insurmountable structural changes, e.g. formation of irradiation-induced defects in the material.

\begin{figure}  
\includegraphics[width=\textwidth]{figures/figure3_4}
\caption[Photocurrent in Log scale.]{Photocurrent through junction in Log scale.
\label{fig:fig3_4}}
\end{figure}

Then, a detailed structural study of the junction was conducted using HRTEM imaging and electron diffraction analysis. The HRTEM images of a representative CdS/p-Si axial heterojunction are depicted in Figure \ref{fig:fig3_2}. These confirm that both nanowires are perfectly-structured single-crystals. The contact interface between two nanowires is atomically smooth. A thin transient graphitic layer is noticed between the wires. It has the origin similar to that mentioned above. In Figure \ref{fig:fig3_2}a, the left-hand-side panel shows the CdS nanowire, whereas the right-hand-side depicts the B-doped Si nanowire. Figures \ref{fig:fig3_2}b and \ref{fig:fig3_2}c present the entire crystallography of the CdS branch. Figure \ref{fig:fig3_2}d and \ref{fig:fig3_2}e present the crystal lattice of the constituting Si nanowire. Figure \ref{fig:fig3_2}f is the fast Fourier transform pattern of Figure \ref{fig:fig3_2}a. Figure \ref{fig:fig3_2}g is the selected area diffraction pattern taken at the junction interfacial region. After the complete structural analysis, which confirmed that high-quality axial heterojunction, constructed of two defect-free pure nanowire single crystals, had indeed been prepared, electrical and optoelectronic tests on it were promptly started. 

\begin{figure}  
\includegraphics[width=\textwidth]{figures/figure3_s2}
\caption[TEM and currents of a Si NW]{(a) In situ alignment of a single B-doped Si nanowire for photocurrent measurements. (b) Dark current and photocurrent from the nanowire. 
\label{fig:fig3_s2}}
\end{figure}

While applying a bias and illuminating the preformed CdS/p-Si heterojunction with a light through the optical fiber threaded inside the holder, dark current and photocurrent data were collected. Figure \ref{fig:fig3_3} and Figure \ref{fig:fig3_4} illustrate a typical current – voltage diagram of a junction illuminated with lasers of 4 different wavelengths. The powers of all laser diodes were fixed to be the same, 13 mW. A photocurrent from the junction at 405 nm was larger than that at 488 nm, and much higher than those at 638 nm, and 808 nm, and the dark current. Because laser diode wavelengths of 405 nm, 488 nm, 638 nm and 808 nm correspond to energies of 3.06 eV, 2.54 eV, 1.94 eV and 1.53 eV, respectively, while CdS and Si nanowire band gaps are ~2.4 eV and ~1.5 eV, respectively, \cite{Fabbri2014}, the heterostructure light absorption at 3.06 eV and 2.54 eV must be easier than that at the energy below 2.4 eV. The higher photoresponse at 3.06 eV than that at 2.54 eV could reflect a complex band diagram of the heterostructure which gives more light absorption possibilities at the higher energy, and hence results in more photo-induced carriers. 


The I-V curves of the CdS/p-Si axial junction, where CdS is considered to be an n-type semiconductor (due to S vacancies), and B-doped Si as a p-type semiconductor, did not display an ideal p-n junction parameters. In the forward bias regime, below 10 V, the I-V curves reveal small currents, and the currents become saturated at a bias higher than 20 V. In the reverse bias regime, the currents also exhibit a saturation tendency, over 30 V. 

\begin{figure}  
\includegraphics[width=\textwidth]{figures/figure3_5}
\caption[Photocurrent at different power]{Saturation of the photocurrent and dark current at different levels of light intensity. (a) Current–voltage plots at various laser powers; the inset shows the low-magnification TEM image of the structure under the measurements; the colored arrow illustrates the light incidence; (b) current-laser power plots at the two selected bias voltages.
\label{fig:fig3_5}}
\end{figure}

To compare, for an individual  CdS nanowire or a B-doped Si nanowire, saturation did not occur up to 10 V, as marked in Figure \ref{fig:fig3_3} and Figure \ref{fig:fig3_s2}. Once a bias larger than 10 V had been applied to a single CdS or Si nanowire, their structural breakdown readily happened. This was due to Joule heating at a high current density. \cite{Wu2004} In my experiments, the single crystalline nanowires (having narrow contact areas with the electrodes) were particularly vulnerable to current densities larger than ~104 $A\cdot cm^{-2}$. Therefore, it is apparent that the CdS/p-Si junction effectively hampered the current density at a high bias, and thus protected the structures from damage. 


Test experimental runs were performed to understand the associated factors responsible for the current saturation. As marked in Figure \ref{fig:fig3_5}, by recording  photocurrents for different light intensities at 488 nm, we observed that photocurrents and dark current of the CdS/p-Si axial nanowire junction at changing laser powers were saturated accordingly.The saturated currents were proportional to the laser diode power. Such phenomenon implies that the CdS/p-Si junction transfers light intensity into an electrical signal with the excellent voltage tolerance. As shown in Figure \ref{fig:fig3_s3}, another junction with a smaller contact area, which was illuminated by a 405 nm laser at 13 mW, also demonstrated a profound saturation effect. The photocurrent saturated at 5 V, the saturation current values were 15 nA and –75 nA. Therefore, it is likely that the junction contact conditions largely affect the saturation threshold and the current value. Generally, the photocurrents of nanowire photodetectors are proportional to the voltage \cite{577926470}, while, at the same time, they are proportional to the light intensity. This means that the voltage should be certain and stable within a limited tolerance for reliable light detection. Consequently, the presently built CdS/p-Si axial nanowire junctions show a great promise toward light intensity sensing not only because they can restrict the current density at a high voltage, but also because they do not need an entirely stable voltage. 

\begin{figure}  
\includegraphics[width=\textwidth]{figures/figure3_s3}
\caption[Another junction]{ (a,b) TEM images, and (c) SAED pattern of a CdS/p-Si junction having the narrow contact region. (d-e) Photocurrent measurements on this junction. 
\label{fig:fig3_s3}}
\end{figure}

The observed phenomenon of photocurrent saturation at the high bias levels seems to to be analogous to that discovered for a planar Metal-Semiconductor-Metal (MSM) photodetector \cite{577926472}, electron beam excited CdS single crystals \cite{Dervos2004}, a model p-n junction \cite{577926474} and a p-n junction solar cell \cite{Gu2005}. Firstly, the electron beam excitation mechanism should be excluded, because the electron beam was shut during the measurements, and it did not have a continuous effect on a photocurrent of the material \cite{Dervos2004}. Secondly, the as-prepared CdS nanowires were confirmed to be defect-free single crystals with a marginal number of sulfur vacancies, and, hence, they did not operate as a heavily doped donor. For an ideal p-n junction, the dark current could be written as: 
$$I=I_s\left(e^\frac{V_D}{nV_t}-1\right)\eqno{(1)}$$

Equation ($1$) is named as Shockley’s diode equation \cite{577926477} and the dark current density of a non-ideal diode could be expressed as: 
$$J_F\approx-\frac{q(2D_p)N_i}{L_p}e^\frac{qV}{mk_0T}\eqno{(2)}$$
The factor $m$ in Equation ($2$) is changeable. Under circumstances of very low and very large biases, $m = 2$, and $J_F\propto e^\frac{qV}{k0T}$, the recombination current or the high injection takes an effect, current densities increase linearly; when bias is on a medium level, $m=1$, $J_F\propto e^\frac{qv}{2k0T}$, the diffusion current becomes important, and the current density exponentially increases. This is similar to the observed I-V curve trends, but Equation (2) does neither exactly explain the photocurrent saturation phenomenon nor the relationship between the saturation current value and incident light intensity. From the theory of photocurrent saturation developed by Mohammad and Abidi \cite{577926474}, for lightly degenerated semiconductors, where spatial variations of dielectric constant, effective mass, carrier lifetime, mobility and diffusivity are significantly small and could be neglected, the total current could be expressed as: 
$$I=Aqg\left ( L_{n}^{*}+L_{p}^{*} \right )-\frac{\left ( e^{\frac{qV_j}{kT}-1} \right )\cdot \left (  I_0+I_{0}^{'} e^{-z} \right )}{1-e^{-2z+z_1+z2}}\eqno{(3)}$$
where, 
$$I_0=\frac{qA}{\tau_a}(n_0(x_{p2})L_n^*+p_0(x_{n2})L_p^*)$$
$$I_0=\frac{qA}{\tau_a}(n_0(x_{p2})L_p^*e^{z2}+p_0(x_{n2})L_n^*e^{z1})$$
$$L_n^*=L_p^*=L_a=\sqrt{D_a\tau_a}$$
$D_a$, $\tau _{a}$ and $g$ are defined as ambipolar-diffusion coefficient, ambipolar lifetime and amibipolar-carrier generation rate in ref [33], respectively. We do not neglect injection, $V_j, V_d$, and  it is considered that $z_1=z_2=0$, $z=\frac{q}{kT}(V_d-V_j)$ in nondegenerated semiconductors with uniform doping, Equation (3) could be written as: 
$$I=2qAg\sqrt{D_a\tau_a}-k_v(n_0x_{p2}+p_0x_{n2})\eqno{(4)}$$
$k_v$ is defined as a factor to simplify the equation. The first term in the right-hand side of this formula represents the uncompensated current relative to light intensity, and the second term expresses the reduction of this photocurrent owing to spatial dependence of band structure of the junction and the junction potential produced by high injection. For a single B-doped Si or a defect-free CdS nanowire, the current densities increase with bias, as these do for a normal semiconductor nanowire with Schottky contacts. However, for the CdS/p-Si nanowire junctions with a limited junction area and under a large bias, the second term of Equation (4) becomes insignificant, and, therefore, the photocurrent does not increase with a bias but does with the light intensity. 
The observed saturation current values in correspondence with incident light intensities could be explained based on several factors. Because a sufficient bias must be applied to have the flat band at the anode and separate the generated carriers, after the threshold bias, the photocurrent started to notably rise, but when the bias is large enough, effective carriers generated by the incident light become saturated for transmission. In addition, we claim that the CdS/p-Si axial nanowire heterostructures are particularly sensitive to three factors: the relative sizes of the two building blocks (this affects carrier mobility), carrier density and light absorption efficiency; interface crystallography (which also affects mobility), junction parameters; and light intensity; which influences the photocurrent saturation value. 

%table with figure number
\begin{table}[ht]
\centering
\begin{tabular}{|c|c|c|c|c|c|}
\hline
Experiment No. & 1 & 2 & 3 & 4 & 5\\
\hline
CdS NW Length ($\mu$m) & 3 & 0.8 & 1.6 & 10 & 13\\
CdS NW Diameter (nm) & 240 & 38 & 135 & 187 & 120\\
Si NW Length ($\mu$m) & 0.9 & 1.5 & 1.3 & 0.2 & 1\\
Si NW Diameter (nm) & 55 & 44 & 44 & 46 & 63\\
Photoresponse detected & 26/26 & 22/22 & 16/16 & 57/57 & 16/16\\
Saturation current positive (nA) & 16 & 0.50 & 3.2 & 0.12 & 0.16\\
Saturation current negative (nA) & -27 & -0.69 & -7.2 & -0.19 & -0.2\\
\hline
\end{tabular}
\caption[Reproducibility of the saturation effect]{Five independent CdS/p-Si nanowire axial heterojunctions exhibiting the saturation effect all having varying values of sizes and currents. 
\label{table:3_1}}
\end{table}

Over this work I fabricated and thoroughly tested 5 distinct CdS/p-Si nanowire junctions. All of them exhibited the regarded saturation effects with somewhat varying parameters, as depicted in Table \ref{table:3_1}. The results imply that the saturation effect is natural and highly reproducible during \emph{in situ} TEM. 

\section{Conclusions}
To sum up, an original {\em in situ} HRTEM technique to construct individual axial nanowire junctions (perfectly emphasizing the modern {\em nanoarchitectonics} concept) has been demonstrated for the first time. 
{\em In situ} HRTEM and in-tandem structural characterizations and optoelectronic tests highlight the photosensing properties of the single-crystalline axial CdS/p-Si nanowire junctions. The junctions exhibit good selectivity toward the light frequencies higher than those of the yellow range. The junctions possess a specific photocurrent saturation effect; this could be utilized in low-consumption light intensity sensing and integrated tunable voltage-driven applications thanks to the corresponding current limitations and excellent tolerance toward possibly unreliable and unstable bias. \\
The developed nanoarchitectonics-based approach employing \textit{in situ} structural design and measurements gives a strong motivation for establishing new operational principles of single crystal nano-devices. 
Furthermore, it is also envisaged that the near-field scanning technique could be also integrated within the designed system for even better understanding of the exciting nanoscale optoelectronic phenomena.\cite{Gu2005,Xiang2012}.



