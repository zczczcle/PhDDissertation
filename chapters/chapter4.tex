%update: Jan 15 fixed grammar according to prof notes
%update: Jan 09-11 prof check
%update: Jan 03 rewrite all. 
%update: Dec 29 rewrite two parts. 
%update: Nov 21 citation done , figure done. 
%update: Nov 10 by zc, add more text(slightly edited), uploaded and compiled all figures. 
%update: Nov 09 by professor, rewrite the first half of text. 

%\begin{savequote}[75mm] 
%I am somewhat exhausted; I wonder how a battery feels when it pours electricity into a non-conductor?
%\qauthor{Sherlock Holmes (Arthur Conan Doyle)} 
%\end{savequote}

\chapter{\emph{In situ} TEM Electrical Probing for Ultrastable Sodium Ion Batteries}

\newthought{Sodium-ion batteries (SIBs)}, as an important alternative for future energy storage, have been on stage since 1980s. Among many anode materials, elemental phosphorus (P) has attracted most of the interest in recent years because of large theoretical capacity, i.e. 2596 mAh/g. The prime disadvantage, however, of a P anode is its poor conductivity and fast structural degradation due to volume expansion, as much as >490\%, over working cycles. To address this issue, I redesigned the anode structure \textit{via} fabricating a flexible paper made of amorphous P and N-doped graphene. The as-fabricated anode delivers ultrastable characteristics and superb rate capability, 809 mAh/g at 1500 mA/g. The extraordinary structural integrity of this new anode was then studied \textit{via in situ} experiments inside a high-resolution transmission electron microscope (HRTEM), thereby the cyclic dynamics and sodiation/desodiation mechanisms were thoroughly understood. To confirm my experimental results, Density Functional Theory (DFT) calculations were additionally performed to indeed confirm that the N-doped graphene contributes to an increase in capacity for Na storage, and to an improved anode rate performance.

\section{Introduction}
Sodium ion batteries (SIBs) are receiving considerable attention and have bright expectations as one of the most promising alternatives to lithium ion batteries (LIBs) for energy storage.\cite{Ren2014,Yang2011c,Liu2014a,Wen2014b,Shen2015b,wang2014e,Wu2014b,Yao2015b,Ni2014b} Both battery types possess analogous chemistry, but SIBs are cheaper because of more abundant sodium natural resources. In the recent reports, the cathode performance in SIBs was found to be comparable with that of the LIBs.\cite{Sun2014b,Barpanda2014b} However, the performance is still not good enough for the immediate practical applications. There have been an increasing interest in developing high-power and high-capacity anode materials for the next generation SIBs,\cite{Ong2011b,Palomares2012b,Zhu2014b,Yu2014b,Berthelot2011b,Qian2012d,Wang2013g,Komaba2011b,Cao2012b,Wang2013h,Xu2013b,Qian2012e} such as transition metal oxides\cite{Zhu2014b,Yu2014b,Berthelot2011b}, Prussian blue analogues\cite{Qian2012d,Wang2013g}, hard carbon materials\cite{Komaba2011b}, nanowires\cite{Cao2012b}, graphene\cite{Wen2014b,Wang2013h}, tin composite\cite{Xu2013b} and antimony-based materials\cite{Qian2012e}, but their specific capacities are still not competitive for future applications(<800 mAh/g). \\

\begin{figure}  
\includegraphics[width=\textwidth]{figures/figure4_s1}
\caption[Peeling off after cycling]
{Schematic diagram of the structural fracture of a high-volume-change type P anode, in which P nanocrystals anchored on the graphene layer are placed on the surface of a Cu current collector. After 1st discharging, there is a large volumetric expansion (>200-500\%) for P NCs; during cycling, such large volume change will lead to the pulverization and thus peeling off the electrode.
\label{fig:4_s1}}
\end{figure}

Among the hottest materials, elemental phosphorus (P) is one of the most attractive candidates with an ultra-high theoretical capacity of 2596 mAh/g\footnote{All capacities mentioned though the thesis are calculated based on the weight of composites.},\cite{Qian2013b,Kim2013c,Li2013c} i.e. about seven times higher than that of the commercial graphite anode in LIBs. The key challenge associated with a phosphorus anode is its rapid structural degradation caused by huge volume change (>490\%) under cycling. For conventional crystalline phosphorus (Figure \ref{fig:4_s1}), upon Na insertion, the P crystals are pulverized and thus the electrode film surface cracks (as a result of ~200-500\% volumetric expansion), leading to phosphorus peeling off from the current collector, which, in turn, gives a significant performance fading. So, stabilizing or sustaining the rigidness of the phosphorus anode structure during cycling is the practical key toward the improvement of the cycling performance of a P-based SIB anode. Very recently, notable breakthroughs have been witnessed in stabilizing anode structure through a design of amorphous P/carbon hybrids.\cite{Qian2013b,Kim2013c,Li2013c} For example, Qian et al. reported that the amorphous P/C hybrids prepared under high-energy mechanical milling had demonstrated a considerable capacity retention of 68\% after 60 cycles at the current density of 250 mA/g.\cite{Qian2013b} Kim et al. demonstrated that a similar amorphous P/C structure is able to deliver as high as 80\% reversible capacity and less than 7\% capacity fading after 30 cycles at a current density of 143 mA/g.\cite{Kim2013c} In addition, Chou et al. fabricated a composite anode \textit{via} simple hand-grinding of commercial microsized red phosphorus and carbon nanotubes (CNTs); this demonstrated a high capacity retention of 76.6\% over 10 cycles.\cite{Li2013c} The improvements in cycling stability in the regarded reports were indeed remarkable, however, capacity retentions of less than 80\% still indicate that further developments are still needed in order to meet the practical requirements.\cite{Luo2015b}\\

\begin{figure}  
\centering
\includegraphics[width=300pt]{figures/figure4_s2}
\caption[TEM of GN]
{HRTEM image of an N-doped graphene (GN).
\label{fig:4_s2}}
\end{figure}

To obtain a stable P-based Na-ion battery anode having high capacity retention and rate performance, a flexible hybrid amorphous P-embedded N-doped graphene paper was designed and studied in my work. Unlike the standard methods, which assemble the P and C components by any mechanical mixing (milling or grinding), a smarter design should tackle the existing problems, such as electrode chemical stability during sodiation-desodiation and mechanical robustness after hybridization. Therefore, a few-layered N-doped graphene (GN) (Figure \ref{fig:4_s2}) is herein selected as a substrate, whose two-dimensional (2D) nanosheet architecture provides a decent framework ensuring uniform deposition of amorphous P.\cite{Nicolosi2013b,Huang2015b} Several advantages associated with the designed amorphous P@GN hybrids (Figure \ref{fig:4_1}a) prepared by the developed so-called “phase-transformation” route are:\\

\begin{figure} [ht]
\centering
\includegraphics[height=0.8\textwidth, angle=-90]{figures/figure4_1}
\caption[Design and characterizations of P@GN]
{(a) Illustrative scheme of the designed layered anode structure. (b) SEM image of the cross section of a P@GN paper, the inset shows its paperlike appearance. (c) HRTEM image and the corresponding FFT pattern of the P@GN portion confirming its amorphous structure. (d-e) P2p and N1s XPS spectra of P@GN. N2 and N3 represent pyrrolic N and pyridinic N, respectively. 
\label{fig:4_1}}
\end{figure}

\begin{itemize}
	\item[i] Compared with crystalline P (Figure \ref{fig:4_s3}), amorphous P (Figure \ref{fig:4_2}) is more stable because of its relatively small volume change.\cite{Qian2013b,Kim2013c} Through the uniform confinement of the amorphous P within GN frameworks, the flexible GN can effectively buffer the volume change. This effectively prevents the electrode fracture and ensures the improvement of the battery capacity retention during electrochemical cycling;\\
    \item[ii] The possibly formed robust P-C bonds between P and GN layers anchor both components, serving like several elastic “springs” between them; this also helps to further enhance the stability of the anode;\\
    \item[iii] The GN nanosheets provide high conductivity electron transport networks and robust mechanical backbones, so that amorphous P could be very electrochemically active. In addition, the high tenacity of GN is useful to accommodate the volumetric expansion of P without mechanical damage or peeling off effects. Furthermore, N-doped graphene also contributes with a certain capacity to the SIBs and brings the fast sodium ion transport according to the DFT calculations.

Thus in this Chapter, I show that the above-mentioned three key features of the amorphous P@GN structure endows the large-volume-change anode with a superb capacity retention (>85\% over 350 cycles), outstanding cycle stability (0.002\% decay per cycle from 2nd to 350th cycle), and excellent rate capability (809 mAh/g at 1500 mA/g). Most importantly, state-of-the-art {\em in situ} probing experiments in HRTEM and supporting theoretical calculations finally uncover the key advantages of the present design and ensure the future developments of the P-based high-performance SIB anode structures, while getting deep insights into the associated atomistic mechanisms. 
\end{itemize}

\begin{figure} 
\centering
\includegraphics[width=\textwidth,angle=0]{figures/figure4_s3}
\caption[Design and characterizations of PNCs@G]
{a) Schematic illustration of an individual P nanocrystals@G (P NCs@G) nanosheet prototype sodium battery device fabricated under {\it in situ} TEM. b) TEM image of the nano-SIB at the initial stage. c-d) Time-dependent TEM images and SAED patterns of sodiated P NCs@G nanosheet upon sodiation at 5 s and 120 s. e) The schematic illustration of the 1st sodiation-desodiation process of P@GN nanosheet. Red arrows indicate the shrinkage, while blue arrows indicate particles peeled off. Scale bars: 200 nm. 
\label{fig:4_s3}}
\end{figure}

\section{Experimental}
%from main text.
Amorphous P@GN paper was prepared by a designed "phase-transformation" route. Bulk red P was heated to form P4 vapors in a sealed ampule, which were adsorbed and deposited within the inter-layers of GN; it changed back into amorphous red P after condensation.\cite{Roth1947b} Thus, a butter-bread-like structure composed of flexible conductive GN layers and thin amorphous red P layers between them was fabricated (Figure \ref{fig:4_1}b). 
%from SI.

\begin{figure}  
\centering
\includegraphics[width=320pt]{figures/figure4_s4}
\caption[TGA curve of P@GN]
{TGA curve of P@GN sample.
\label{fig:4_s4}}
\end{figure}

Synthesis of P@GN: \\
Graphene oxide (GO) suspension used in this work was prepared \textit{bia} a modified method. 50 mg of graphene oxide was loaded in a ceramic boat in a tube furnace followed by its heat treatment at 600 °C for 1.5 h in a gas mixture of NH3 and Ar (1:2 vol/vol) under a total flow rate of 150 ml/min. Commercial red phosphorus was dispersed in water and put into a Teflon-lined stainless autoclave. The autoclave was heated to 200 °C and maintained for 12 h to remove surface oxides. As-prepared N-doped graphene (GN) products were properly mixed together with the pretreated red phosphorus powder, and sealed in an ampule in an inert Ar atmosphere. The sealed reactors were thermally treated at 500 °C for 1 h and then at 280 °C for several hrs in the furnace, before natural cooling to room temperature. The final product was washed with CS2, water and methanol, and then dried at 60 °C.\\

Characterization: \\
TEM images were taken with a 300 kV JEOL 3000F microscope. Samples were first dispersed in ethanol and then collected using carbon-film-covered copper grids. To avoid possible electron beam effects (such as radiolysis or sputtering damage of both Na-containing species and graphene lattice) the beam intensity was minimized. Scanning electron microscopy (SEM) images were recorded on a Hitachi S4800 electron microscope operating at 15 kV. XRD patterns were taken on a Philips X Pert PRO MPD X-ray diffractometer operated at 35 kV and 45 mA with Cu Kα radiation. XPS measurements were carried out on an ESCALab220i-XL spectrometer by using a twin-anode Al Ka (1486.6 eV) X-ray source. All the spectra were calibrated to the binding energy of C 1s peak at 284.6 eV. The background pressure was ~3 x 10-7 Pa. Raman spectra were collected on a Horiba Jobin-Yoon T6400 Raman spectrometer.\\

Electrochemical tests: \\
The electrochemical properties of P@GN and P NCs@G samples were studied using a 2032-type coin cell on a Hokudo Denko Charge/Discharge apparatus. The working electrode was prepared by directly pressing a piece of sample onto the Cu mesh current collector. Na metal foil was selected as the reference and counter electrode. The electrolyte was 1 M \ce{NaPF6} in ethyl carbonate (EC) and diethyl carbonate (DEC) ($EC : DEC = 1 : 1 vol/vol$). The cells were assembled in a glove box filled with a pure argon gas.\\ 

Construction of individual prototype P@GN (P NCs@G)-based SIB: \\
\textit{In situ} TEM observations were conducted in a JEOL-3100 FEF equipped with an Omega filter and a {\em Nanofactory Instruments} STM-TEM holder. In order to build up the test cell, an individual P@GN or P NCs@G nanosheets were attached to the fresh-cut flattened Au wire, which was then attached to the piezo-manipulator. A small piece of Na foil was placed to another Au wire as a reference and counter electrode. Before insertion of the holder into the TEM column, a piece of Na foil covered with a Na2O layer was placed on the surface of metal Au tip. Then isolated P@GN or P NCs@G samples were chosen for direct electrochemical tests. Under the following delicate piezo-driven TEM mechanical manipulations the two electrodes were connected and the probe cell was finally constructed. The sodiation was carried out at a negative bias in the range of -2 V to 0 V with respect to the Na metal.\\

DFT calculations:\\
The first principle calculations were carried out using the Vienna ab initio simulation package (VASP),\cite{Kresse1996} where projected-augmented-wave (PAW) potential was adopted.\cite{Kresse1999} The functional of Perdew, Burke, and Ernzerhof (PBE) and the generalized gradient approximation (GGA)\cite{Perdew1996} were employed in the calculations. We used a $3\times3\times1$ mesh in the irreducible Brillouin zone for structure relaxation and $6\times6\times1$ mesh for self-consisted calculations. In all the calculations the forces were relaxed to the values lower than 0.02 eV/Å.

\section{Results and discussions}

%text above is edited by prof on Nov9. 
%the following part is added on Nov10, which is rephrased by zc on Dec 29.

It is considered that some P-C bonds possibly exist between amorphous P and GN layers which would tightly anchor those layers. Scanning electron microscopy characterization (Figure \ref{fig:4_1}b) illustrates the cross section of the structure of anode, where we can see a lot disordered and flexible layered structures. It is considered that most GN sheets can not be tranfered back to graphite by restacking even under longtime heating or mechanical induced compression.\cite{Huang2015b} From HRTEM image and the corresponding fast Fourier transform (FFT) patterns, crystallography of P was uncovered: it is amorphous, as depicted in Figure \ref{fig:4_1}c. We were able to observe the disordered nature of P@GN of structure from the diffused rings and the absence of clear lattice fringes. 

\begin{figure}[ht]
\centering
\includegraphics[width=0.8\textwidth]{figures/figure4_s5}
\caption[C1s XPS spectra comparison]
{a-b) C1s XPS spectra of PNCs@G and P@GN samples. c) Overlay of the two surveys for C1s spectra of the two samples. 
\label{fig:4_s5}}
\end{figure}

The constitution of P in the composite was characteried to be ~66\% by thermogravimetric analysis curve as shown in Figure \ref{fig:4_s4}. In Figure \ref{fig:4_1}d, the P2p X-ray photoelectron spectroscopy (XPS) reveals 2p1/2 and 2p3/2 doublets, where two peaks at 129.75 eV (2p3/2) and 130.6 eV (2p1/2) suggest the possible existance of P-C bonding.\cite{Jiao2014b,Niu2014b,Zhang2013b} 

According to the theoretical simulations,\cite{Sun2014b,Claeyssens2009b} among all P-C bonding types -- sp3, sp2 in plane, sp2 at edge, and sp2 in the aromatic ring -- the most stable one is the sp2 hybridized P-C bonds in aromatic ring because of bond length in the \pi-p* conjugation plane is the shortest. As the inset image illustrates in Figure \ref{fig:4_1}e, GN would provide sp2 P-C bond at the edge and/or at the aromatic rings. \\

In addition, another evidence from C1s XPS spectra, as shown in Figure \ref{fig:4_s5}, depicts the possibly existing P-C bonds in P@GN. In order to get clear comparison, P nanocrystlas in pure graphene sample (this sample name will be called as P NCs@G) was also prepared for test. As shown in Figure \ref{fig:4_s5}a, clearly, there are no P-C bonds found in P NCs@G. 

As compared with the sample set of P NCs@G, the sp2 carbon atom fraction decreases while sp3 C-C (285.3 eV) bonds prevail in P@GN (Figure \ref{fig:4_s5}b) possibly due to some defects caused by nitrogen doping. Three N-doping types are found to exist in P@GN (Figure \ref{fig:4_1}e): graphitic, quaternary N (N1, 401.7 eV), pyrrolic N (N2, 400.2 eV), and pyridinic N (N3, 399.1 eV).\cite{Roth1947b,Wang2012e,Wang2014f,Wang2013i} The N2 and N3 dopants are generally thought to be located at the edges or surface defect sites such as vacancies.\\

\begin{figure}  [ht]
\includegraphics[width=\textwidth]{figures/figure4_s7}
\caption[Elemental maps of P@GN]
{
HAADF-STEM image, and C-, P- and N-elemental maps of a P@GN nanosheet. 
	
\label{fig:4_s7}}
\end{figure}

The specially designed layere structure exhibits excellent electrochemical performances. In Figure \ref{fig:4_2}, the battery performances of the P@GN were evaluated by using standard CR2032 coin cells. 

As a comparison sample set, red P nanocrystals (NCs) placed on pure graphene (P NCs@G) were also manufactured by grinding of nanoscale red phosphorus and graphene. As shown in Figure \ref{fig:4_2}a, the initial Coulombic efficiency of P@GN is 87\%, higher than that of 85\% for a reported P@C hybrid electrode.\cite{Li2013c} 
From the 2nd to 120th cycles at 200 mA/g or 350th cycle at 800 mA/g, the Coulombic efficiencies are more than 98\%. A discharge plateau, as depicted in Figure \ref{fig:4_2}b, corresponds to an anodic peak spreads from 0 V to 0.5 V, implying the formation of \ce{Na3P} with a theoretical capacity of 2596 mAh/g. The conversion chemical reaction from P to \ce{Na3P} takes place. 
The chemical reaction process is then eventually proved by {\em in situ} HRTEM experiments which will be discussed in the following paragraphs. In a reversed scan, a main anodic peak appears at 0.53 V, which possibly matches the main sodium ion de-intercalation process \textit{via} chemical reaction from \ce{Na3P} to P. However, no clear peaks at any other potentials (for instance 0.63 V assigned to NaP) were found. This indicates that this electrode probably experiences a reversible discharging/charging chemical cycling between \ce{Na3P} and P and affords very high capacity because of existence of \ce{Na3P}. \\

Over 350 battery cycles, the capacity retention of P@GN battery is more than 85\%. From the 2nd to 350th cycle at 200 mA/g, the capacity decay is even less than 3\%. 
The excellent capacity retention and stability (about 0.002\% decay per cycle) are among the best cycling stablity performances of all reported P-based anodes to date. 
To reveal the mechanism of the unusual cyclic stablity, the original and after-cycled P@GN electrodes were examined by a high angular annual dark field (HAADF) imaging in a scanning TEM (STEM) mode with the EDS mapping as illustrated in Figure \ref{fig:4_2}c, Figure \ref{fig:4_s7} and Figure \ref{fig:4_s8}. 
Obviously, the integrity of the battery structure is maintained quite well in the desodiated states after even 120 cycles. 

\begin{figure}  [ht]
\includegraphics[width=\textwidth]{figures/figure4_s8}
\caption[Elemental maps of P@GN after 150 cycles]
{
The HAADF-STEM image and the corresponding elemental maps of a P@GN nanosheet at the fully desodiated state after 150 cycles.
	
\label{fig:4_s8}}
\end{figure}


\begin{figure}  
\includegraphics[height=\textwidth,angle=-90]{figures/figure4_2}
\caption[Performance of P@GN SIB]
{
(a) Cyclic capacity and Coulombic efficiency of P@GN at 200 mA/g and 800 mA/g. (b) Galvanostatic charge and discharge profile of P@GN anode at 200 mA/g. (c) The HAADF-STEM image and the corresponding EDS maps of a P@GN nanosheet at the desodiated state (after 120 cycles). (d) Rate capabilities of P@GN and P NCs@G. (e) Nyquist plots and equivalent circuit model of P@GN and P NCs@G electrodes after 10 cycles at 0.1 A/g in the discharged state. 
\label{fig:4_2}}
\end{figure}

It is also noted that upon sodiation, distribution of sodium species is very homogeneous over the nanosheet (Figure \ref{fig:4_2}c), this suggests the successful intercalation of sodium ions. 
The reduced volume expansion of P layers are significantly buffered and confined by GN layers, which is the key for preventing failure of the anode structure. 
Moreover, P@GN anode material exhibits improved rate capability as compared with P NCs@G material, as marked in Figure \ref{fig:4_2}d. 
At a high current rate (1500 mA/g), the reversible capacity still reaches 809 mAh/g for P@GN. This capacity is twice higher than that of the theoretical capacity of the commercial graphite (370 mAh/g) in LIBs. And, of course, these results are far better than those of the P NCs@G (10 mAh/g at 1500 mA/g). \\
This is especially meaningful for future secondary battery choice. In fact, sodium-ion batteries are expected to be of low cost and high capacity. 

To compare the battery anode kinetics of P@GN and P NCs@G, their electrochemical impedance spectroscopy (EIS) was performed as ploted in Figure \ref{fig:4_2}e. The Nyquist curves demonstrate that a diameter of the semicircle for P@GN anode material in high-medium frequency region is much smaller than that of P NCs@G electrode. This suggests that P@GN anode possesses a lower contact and better charge-transfer impedance. Based on the modified Randles equivalent circuit, shown in the inset of Figure \ref{fig:4_2}e, the P@GN anode exhibits a significant lower charge-transfer resistance. Therefore, P@GN holds a high electrical conductance and also provides more stable surfaces such as SEI layer. This leads to the better rate capability and reversible capacity as comparison with P NCs@G. 
In addition, the angle of low-frequency slope for P@GN (76.8 degrees) is steeper than that of P NCs@G (49.1 degrees), indicating higher diffusivity of \ce{Na+} for sodium ion uptake and extraction in P@GN anode due to the steep low-frequency tail.\cite{Sun2014b} \\

\begin{figure}  
\centering
\includegraphics[width=300pt,angle=0]{figures/figure4_3ab}
\caption[{\it In situ} probing on P@GN SIB setup]
{
(a) Schematics of an individual P@GN nanosheet sodium ion battery device test performed by {\em in situ} TEM. 
(b) TEM image of the SIB at initial stage. Scale bars: 100 nm.
\label{fig:4_3ab}}
\end{figure}

It is very important to maintain the structural integrity during cycling to realize stable performance for large capacity and low-cost anode material.\cite{Liu2014a} 
To analyze a mechanism of the anode stability of our ultra-stable decice, I performed an {\em in situ} TEM study of the chemical and structural changes of the as-fabricated anode during electrochemical cycling (Figure \ref{fig:4_3ab}). The {\em in situ} TEM experimental set-up is very similar to some previous reports (Figure \ref{fig:4_3ab}a).\cite{Wang2014f,Wang2012g} \\
The setup mainly consists of two parts: the sample is placed on a gold wire tip, while another gold wire with a small piece of sodium is put on the opposite side. Special care is required for sodium loading. Sodium is loaded to the probe (which is on the TEM holder) in glovebox filled with argon. Then the holder was capped under argon atmosphere. A few seconds before insertion of holder, the cap was removed. 
During transferring process, a very thin layer of \ce{Na2O} was formed on the sodium metal surfaces. The \ce{Na2O} layer serves as a natural solid electrolyte for the single nanostructured SIB. Figure \ref{fig:4_3ab}b depicts a TEM image of a freestanding P@GN nanosheet material. 

Figure \ref{fig:4_3cd} presents the sodiation process of an individual P@GN anode.
A -2 V potential was applied to P@GN anode with respect to the sodium potential.
The anode material immediately expanded in both longitudinal and transverse directions after applying the bias, as shown in Figures \ref{fig:4_3cd}. 
After sodiation, the length of nanosheet increased from initial 210 nm to the sodiated 250 nm, and the length of a regional edge enlarged to 83 nm from the previous 65 nm. 
The experiment further implies that an effective Na transport along/across the hybrid structure indeed took place. 

\begin{figure}  
\centering
\includegraphics[width=320pt,angle=0]{figures/figure4_3cd}
\caption[{\it In situ} sodiation process of P@GN SIB]
{
 TEM image and SAED pattersn of the {\em in situ} SIB at (a) 15 s and (b) 120 s of sodiation during first discharging. Scale bars: 100 nm.
\label{fig:4_3cd}}
\end{figure}

It is noted that the expansion rate in the one direction, from 210 nm to 250 nm, becomes less than that for another one, from 65 nm to 83 nm. This might indicate that small-sized nanoribbons could possess higher electrochemical activities. 
No significant evidence of structural degradation was found even after the entire sodiation (Figure \ref{fig:4_3cd}b). This is confirmed by the decent flexibility of GN and amorphous P layer which can effectively buffer the large volume expansions during insertion of sodium ions. 
The nanosheet thickness also increased from 10 nm to 14 nm, as it is marked by blue color in Figures 3b and 3d during discharge process. 
The SAED pattern of the sodiated P@GN (Figure \ref{fig:4_3cd}) reveals the crystallography information after phase changes after sodiation. 
The main phase of the sodiated anode material is identified to be \ce{Na3P} -- which takes more sodium ions than \ce{NaP} per single P atom. This result is consistent with the battery test shown in Figure \ref{fig:4_2}b. 

\begin{figure}  
\includegraphics[width=\textwidth,angle=0]{figures/figure4_3ef}
\caption[{\it In situ} desodiation process on P@GN SIB]
{
  (a) 5 s and (b) 120 s of time-dependent TEM images of desodiation during 1st charging. Insets show the corresponding schematic atomic structures. The inset in (a) depicts the corresponding SAED pattern. Scale bars: 100 nm.
\label{fig:4_3ef}}
\end{figure}

%this part is rephrased by zc on Jan 2. 
The desodiation process took place when a 2 V bias was applied to the nanostructure. As presented in Figure \ref{fig:4_3ef}, the volume shrinkage became observable along both the longitudinal and transverse directions. 
It is observed that desodiated nanosheet after sodium extraction looks quite similar to the initial status. What is more, the edge segment perfectly kept the initial state. 
The experiment demonstrates that the designed layered structure is able to buffer the anode volume expansion and shrinkage during electrochemical cycling. 
Therefore, the {\em in situ} experiments explain the stable cycling performance as revealed by the {\em ex situ} battery measurements (Figure \ref{fig:4_2}b). 
Integrity of the P@GN nanosheet is preserved quite well, suggesting that the butter-bread-like structure is effective for relaxing the strain and overcoming pulverization caused by volume expansion, and, hence, it can be a very promising anode candidate for SIB.\\

The comparison material, an individual P NCs@G nanosheet, was also build and tested by {\em in situ} microscopy. 
As shown in Figure \ref{fig:4_s3}, during discharging, P nanocrystals expanded immediately. 
The whole P NCs@G nanosheet drastically shrinks instead of expanding upon sodium insertion along all directions, because the sodiated crystals aggregated together and also their size grew larger under the Ostwald ripening. \\

\begin{figure}  
\includegraphics[width=\textwidth]{figures/figure4_4}
\caption[Raman spectra and DFT calculations]
{
(a) Raman spectra of red P, P NCs@G, and P@GN before and after sodiation. 
(b) The schematics of the insertion of sodium into a \ce{C46N3} sheet based on DFT calculations. 
(c-d) DOS of \ce{C46N3} and its sodiated product \ce{NaC46N3}.
\label{fig:4_4}}
\end{figure}

Therefore, the strong adhesive forces between graphene and P NCs compressed the whole structure into an aggregate form of P NCs. 
Note that the phenomenon is more significant at the edges, as compared with the basal plane due to their higher electrochemical activities. 
Moreover, peeling off of an active material can also be observed during {\em in situ} TEM processing which is marked by blue arrows in Figure \ref{fig:4_s3}c-d. Two particles in the upper part and another nanocrystal in the lower part disappeared after sodiation. 
The peeling off of active material implies that the P NCs@G experienced significant expansion, which leads to the irreversible peeling off, and hence, caused a loss of capacity. 
It is noted that some NaP phases are seen in SAED patterns in Figures \ref{fig:4_3cd}, for the desodiated material. The NaP phase (instead of \ce{Na3P}) implies that only conversion of P into NaP rather than into \ce{Na3P} took place. It is know that \ce{Na3P} holds much higher capacity, 2569 mAh/g, than NaP, which is only 856 mAh/gh. 
Sodiation of phosphorus in P NCs@G sample can be associated with its low electrical conductivity and large crystal size. 
Therfore, the irreversible structural failure and the presence of NaP (instead of \ce{N3P}) undoubtedly limit the performance for P NCs@G. 
The {\it in situ} experiment condlusion is consistent with the battery cycling performance and rate properties (Figure \ref{fig:4_2}c). \\
It is believed that maintaining amorphous morphology of P during cycling is a key factor for ultrastable battery performance. 
This is confirmed by the Raman spectroscopy presented in Figure \ref{fig:4_4}a. 
Spectrum features of red P between $300-500 cm^{-1}$ can be attributed to P–P stretching bonds of P9 and P7 cages while establishing the pentagonal tubes in paired layers.\\

Compared with red P material, the amorphous phosphorus and nitrogen doped graphene hybrid structure does not display Raman peaks natural for P, but shows two typical D-band and G-band peaks (peculiar to graphene).\cite{Kim2013c} 
The result indicates that the layerd P@GN paper was well maintained during cycling. 
Additionally, Raman spectra of various samples provide more evidences for the existence of the stable P-C bonds after multiple cycles. 
In Figure \ref{fig:4_4}a, a broad envelope centered at about $700 cm^-1$ for P@GN, which is marked by red circle, could be attributed to P-C bond stretching modes. 
Therefore, stable P-C bonds may exist after cycling, as shown in Figure \ref{fig:4_4}a. \\

Last but not least, DFT simulations were performed to uncover the function of N-doped graphene for battery performance of SIB. 
Some previous works used doped graphenes for LIB and SIB anode materials.\cite{Yang2011c,Wen2014b,Wang2013h,Wang2012e,Wang2014f} 
First, the adsorption energy of sodium ions on different graphene structures were calculated. 
For graphene and N1 graphene, the value is positive, at around +0.50 eV. 
This means that both graphene and N1 graphene are unfavorable for sodium absorption, and, hence, they deliver no capacity to the battery. 
For N2 and N3 doped graphene and P doping, the absorption energy turns to be negative. 
The negative energy value means that the material is attractive for sodium absorption. 
The corresponding sodium absorption energy on GN surfaces are illustrated in Figure \ref{fig:4_4}b. 
Then, the corresponding capacities based on the simulation of the maximum sodium concentration were calculated. 
It is acknowledged that graphene is not conductive, and hence the theoretical capacity is exactly zero.
\ce{C46N3} holds a 0.851 e charge transfer, and capacity of 373.3 mAh/g. This value is 1.2 times of the capacity of hard carbon at 300 mAh/g, suggesting that N and P doped graphenes are decent candidates for modern anode materials due to their positive roles in the electron transport. 

All three kinds of doped graphenes show a large charge transfer from sodium.
For instance, sodium donates 0.853 e charge to \ce{C46N3}. 
It implies that doping sites could provide the high efficiency to improve the interactions between sodium and G surface for future sodium ion storage. 
The high rate capability of the N-doped graphene is also supported by the density of states simulations, as shown in Figure \ref{fig:4_4}c-d, N doping makes \ce{C46N3} metallic. 
As a consequence, theoretically, N doped graphene is favorable for electron and ion transfer, high rate capability, decent capacity, and long battery lifetime for SIB.
\vfill % otherwise looks bad. 

\section{Conclusions}
To sum up, I applied a phase-transformation approach to fabricate nitrogen doped graphene-phosphorus structure with layered morphologies where thin amorphous phosphorus layers are formed within flexible and electrically conductive N-doped graphene structures. 
Advantages of the designed anode material have been studied using various characterization techniques, device tests, {\em in situ} microscopy and theoretical calculations. 
These advantages are namely: \\
(1) Thin P layer on the doped graphene (instead of crystalline P) anode shows ultrastable efficiency of 0.002\% decay per cycle and good rate capability of 809 mAh/g at 1500 mA/g; \\
(2) P-C stable bonds may exist to bond GN and P layers; \\
(3) {\em In situ} HRTEM experiments verified and revealed the reason of the materials' ultrastable performance.\\

%Finally, DFT calculations reveal that doping sites can enhance the interactions between Na and graphene surface, leading to ultrafast sodium energy storage. Our work demonstrates that the designed flexible amorphous P@N-doped graphene structure prepared from a "phase-transformation" approach can greatly improve the cycling and rate performances for future sodium storage. 



